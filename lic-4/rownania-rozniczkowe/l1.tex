\section{Motywacja}

Motywacją do rozważania równań różniczkowych będzie zagadnienie rozpadu promieniotwórczego.
Eksperymentalnie sprawdzono, że tempo rozpadu jest wprost proporcjonalne do masy pierwiastka promieniotwórczego.
Czyli dla pewnej $k>0$ mamy $x'(t)=-kx(t)$, gdzie $x\left(t \right) $ to masa pierwiastka w chwili $t$.
Szukamy funkcji masy zależnej od czasu $m(t)$, pomijać będziemy oznaczenie argumentu, oraz będziemy pisać $x = m$ oraz $\dot{x}:=m'$. 

Zgadnąć możemy, że rozwiązaniem są funkcje postaci $x\left( t \right) = e^{-kt}\cdot x\left( 0 \right) $. Powstaje naturalne pytanie -- czy wszystkie rozwiązania są takiej postaci? W tym przypadku możemy wykonać przejścia równoważne:
\begin{align*}
	\dot{x}+kx&=0\\
	\dot{x}e^{kt}+ke^{kt}x&=0\\
	\left( xe^{kt} \right) ^{\boldsymbol{\cdot}}&=0 
.\end{align*} Zatem $xe^{kt}$ jest stałą, więc $x\left( t \right) =c\cdot e^{-kt} = x(0)\cdot e^{-kt}$. 

\section{Definicja}
Układem równań równiczkowych zwyczajnych $m$-tego rzędu nazywamy wyrażenie: \[
	F\left(t,x,\dot{x},\ddot{x},\ldots,x^{(m)}\right)=0 
,\] gdzie 
\begin{align*}
	x: & \quad\mathbb{R}\supseteq I \to \mathbb{R}^{n}\\
	F:&  \quad\mathbb{R}^{1+\left( m+1 \right) n} \to \mathbb{R}^{k}
.\end{align*}

Zazwyczaj udaje się wyrazić najmniejszą pochodną w postaci rozwikłanej: $x^{\left( m \right) }=f\left( t, \dot{x}, \ldots, x^{\left( m-1 \right) } \right) $. Zwykle zakłada się różniczkowalność lub lipschitzowskość funkcji $f$. 

\section{Redukcja do układu równań pierwszego rzędu}

Jeżeli mamy $x^{\left( m \right) }=f\left(t,\dot{x},\ldots,x^{\left( m-1 \right) }\right)$, to położywszy $y_{k}=x^{(k)}$, otrzymamy $$\dot{y_{0}} = y_1, \ldots, \quad \dot{y_{m-2}}=y_{m-1}, \quad \dot{y_{m-1}}=y_m=x^{\left( m \right) }=f\left( t,y \right) $$ dla $y=\left( y_0,...,y_{m-1} \right) \in \mathbb{R}^{m}$ . 
A zatem mamy $\dot{y} = g\left( t,y \right) $.
\section{Badanie roztworów nasyconych}
Kolejny przykład zastosowania równań różniczkowych. Stwierdzono, że w cieczy można rozpuścić ilość soli, która zwiększa się proporcjonalnie do zmiany temperatur: $\Delta s=k\Delta T$, a więc $\frac{\Delta s}{\Delta T}=kS$, czyli w granicy $\frac{dS}{dt}=kS$. Znamy rozwiązanie tego równania, $S\left( T \right) =S\left( 0 \right) $. Biorąc $s_0=S(T_0), s_1=S(T_1)$, dostaniemy coś, co chętnie bym opisał, ale nie mam notatek. Niech ktoś podeśle. Serio dajcie. :(

\section{Metoda rozdzielania zmiennych}
Załóżmy, że mamy do czynienia z równaniem postaci $x'(t) = g(t)f\left(x(t)\right)$, albo pisząc zwięźlej: $x' = g(t)f(x)$, gdzie $x : \mathbb{R} \supseteq I \to \mathbb{R}$, $f : x\left[I\right] \to \mathbb{R}$. Załóżmy też, że $f(x) \neq 0$ dla wszystkich argumentów. Ponadto, niech dany będzie nam warunek początkowy: $x(t_0) = x_0$. Wprowadźmy następujące oznaczenia:
\begin{align*}
	G(t) =& \int_{t_0}^{t} g(s) ds \\
	F(x) =& \int_{x_0}^{x(t)} \frac{1}{f(s)} ds
\end{align*}
Wówczas równanie $G(t) = F(x)$ zadaje zależność między $x(t)$ a $t$, będącą rozwiązaniem równania. Nie zawsze z takiej zależności możemy 'odczytać' jawny wzór, ale z twierdzenia o funkcji uwikłanej często możemy wnioskować istnienie rozwiązań, choćby lokalnych. Równania postaci $x' = f(x)g(t)$ nazywamy równaniami \textbf{o rozdzielonych zmiennych}. 

\textbf{Przykład}. Rozważmy równanie $e^{-x}(1+x')=1$. Prostymi przekształceniami algebraicznymi sprowadzamy równanie do postaci $x' = \frac{1-e^{-x}}{e^{-x}} = e^x-1$. W tym przypadku $f(x) = e^x-1$, $g(t) = 1$. Stąd wynika, że:
\begin{align*}
	G(t) =& \int_{t_0}^{t} 1 ds = t-t_0+c_1 \\
	F(x) =& \int_{x_0}^{x(t)} \frac{1}{e^s-1} ds = \log|e^{-s}-1|\bigg|_{x_0}^{x(t)} = \log \left| \frac{e^{-x(t)}-1}{e^{-x_0}-1}\right| +c_2
\end{align*}
Zatem $x(t)$ możemy zapisać jako funkcję uwikłaną za pomocą następującego równania:
\begin{align*}
	t-t_0 + C = \log\left|\frac{e^{-x(t)}-1}{e^{-x_0}-1}\right|
\end{align*}
Stała $C = c_1-c_2$ jest dowolna, natomiast zadając stałe $t_0$ oraz $x_0$ sprawiamy, że tak otrzymane rozwiązanie równania różniczkowego będzie spełniać równość $f(t_0) = x_0$. W istocie w tym przypadku możemy wyliczyć $x(t)$ i wyrazić w jawny sposób, jednak nie jest to ani konieczne, ani wygodne.
