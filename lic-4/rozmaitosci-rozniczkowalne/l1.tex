\subsection{Motywacja}
Naukę każdej dziedziny matematyki dobrze jest zacząć od solidnej motywacji, od poznania choćby historycznego rysu zagadnień, którymi będziemy się zajmować. Dzięki temu można uniknąć przeładowania abstrakcją. To, czym będziemy się zajmować na kursie przedmiotów różniczkowalnych, obejmuje zagadnienia naturalnie pojawiające się w wielu działach matematyki czy fizyki. Motywacją dla nas, obiektami które chcemy badać, są następujące obiekty:
\begin{itemize}
	\item powierzchnie
		\begin{figure}[ht]
			    \centering
			        \incfig{torus}
				\caption{Przykładowa powierzchnia drugiego stopnia}
				\label{fig:torus}
		\end{figure}
	\item przestrzenie opisywalne (lokalnie) za pomocą skończonej ilości parametrów, na przykład przestrzenie konfiguracyjne rozmaitych układów fizycznych. W takich układach parametry traktuje się jako stopnie swobodny. Przykładem może być przestrzeń położeń bryły sztywnej, która jest przestrzenią konfiguracyjną o sześciu stopniach swobody - sześciowymiarową rozmaitością.

	\item podzbiory $\mathbb{R}^n$ lub nawet $\mathbb{C}^n$ zadane równaniami algebraicznymi, np. równanie $z_1^2 + z_2^2 + z_3^3 = 1$ w $\mathbb{C}^3$ opisuje czterowymiarową rozmaitość.
\end{itemize}
Dodatkowa motywacja pochodzi z przyczyn fizycznych i matematycznych. W rozmaitych gałęziach nauki w sposób całkowicie naturalny pojawiają się obiekty takie, jak te wyżej wymienione. Z przyczyn praktycznych czy teoretycznych pragnęlibyśmy uprawiać na takich obiektach analizę, np. mówić o różniczkowalności funkcji pomiędzy rozmaitościami, rozpatrywać równania różniczkowe czy analizę wektorową. Możemy np. myśleć o tym, że we wspomnianej wcześniej przestrzeni położeń bryły sztywnej mamy do czynienia z polem wektorowym, które decyduje o tym, jak potencjalnie może się przemieszczać nasza figura.

\section{Rozmaitości topologiczne}
\begin{definition}[Rozmaitość topologiczna n-wymiaropwa]
	Przestrzeń topologiczną $M$, która spełnia następujące warunki:
	\begin{itemize}
		\item jest przestrzenią Hausdorffa ($T2$, tzn. dla każdych $x,y \in M$ istnieją $U, V$ otwarte, takie, że $x \in U$, $y \in V$ oraz $U \cap V = \emptyset$)
		\item ma przeliczalną bazę
		\item jest lokalnie euklidesowa wymiaru n, to znaczy każdy punkt ma otwarte otoczenie homeomorficzne z pewnym otwartym podzbiorem w $\mathbb{R}^n$.
	\end{itemize}
	będziemy nazywać \textbf{n-wymiarową rozmaitością topologiczną}.
\end{definition}

Innymi słowy rozmaitością nazywamy coś, co pod lupą wygląda jak przestrzeń $\mathbb{R}^n$, chociaż gdy spojrzymy z daleka to może być zupełnie dziki obiekt. Można by zadać pytanie: no dobrze, punkt trzeci narzuca nam tę lokalną euklidesowskość, ale co z pierwszymi dwoma? Po co nam są takie wymagania?

\subsubsection{Uwaga pierwsza: Hausdorffowość}
Warunek Hausdorffowości wyklucza niektóre patologie. (\textit{Tutaj pojawi się rysunek, gdy tylko nauczę się robić.}) Ogólniej: Hausdorffowość przerzucić lokalne własności obiektów z $\mathbb{R}^n$ na obiekty w $M$. Np. jeżeli mamy punkt $x \in M$, $\phi : U \to \bar{U}$  gdzie $U$ oraz $\bar{U}$ to otwarte otoczenia punktów $x$ oraz $\phi(x)$ odpowiednio, a ponadto mamy zbiór $\bar{K} \subset \bar{U}$ zwarty w $\mathbb{R}^n$, to warunek bycia przestrzenią Hausdorffa gwarantuje nam, że zbiór $\phi^{-1}\left(\bar{K}\right)$ jest zwarty w $M$. Innymi słowy pozwala lokalnie cofać zwartość.

\subsubsection{Uwaga druga: przeliczalna baza}
Przeliczalność bazy wyklucza np. 'zbyt duże' przestrzenie, np. $\mathbb{R}^n \times \mathfrak{c}$ (continuum rozłącznych kopii $\mathbb{R}^n$). Bez tego warunku nie byłoby następujących własności rozmaitości topologicznych:
\begin{itemize}
	\item każde pokrycie zbiorami otwartymi zawiera przeliczalne podpokrycie,
	\item każda rozmaitość $M$ jest wstępującą sumą otwartych podzbiorów, których domknięcia w $M$ są zwarte,
	\item każda rozmaitość jest parazwarta (tzn. każde otwarte pokrycie posiadania lokalnie skończone rozdrobnienie),
	\item każdą rozmaitość można zanurzyć w $\mathbb{R}^N$ dla odpowiednio dużego $N$.
\end{itemize}

\subsubsection{Euklidesowość wymiaru n}
Można zadać pytanie, czy jeżeli mamy rozmaitość topologiczną n-wymiarową $M$, to czy możliwe jest, że istnieje liczba m taka, że $m \neq n$, że $M$ jest także rozmaitością topologiczną m-wymiarową?

Okazuje się, że nie. Negatywnej odpowiedzi udziela twierdzenie o niezmienniczości obszaru, udowodnione przez L. Brouwera w 1911 roku. Dowiódł on, że dla $n \neq m$ otwarty podzbiór w $\mathbb{R}^n$ nie może być homeomorficzny z otwartym podzbiorem $\mathbb{R}^m$. Stąd liczba $n$ jest jednoznacznie przypisana do $M$ i nazywa się ją wymiarem $M$. Oznaczenie: $\mathrm{dim}(M) = n$.

\subsection{Mapy i atlasy}
\begin{definition}[Mapa, zbiór mapowy]
	\textbf{Mapą} na rozmaitości topologicznej $M$ nazywamy parę $(U,\phi)$, gdzie $U \subset M$ jest otwartym podzbiorem, zaś $\phi : U \to \bar{U} = \phi(U) \subset_{otw} \mathbb{R}^n$ jest homeomorfizmem na otwarty podzbiór $\mathbb{R}^n$. Zbiór $U$ nazywamy \textbf{zbiorem mapowym}. $(U, \phi)$ jest \textbf{mapą wokół punktu} $p \in M$ gdy $p \in U$ oraz $\phi(p) = 0 \in \bar{U} \subset \mathbb{R}^n$. $(U,\phi)$ nazywamy też lokalnymi współrzędnymi na $M$.
\end{definition}

Odnotujmy obserwację: rozmaitość topologiczna jest pokryta zbiorami mapowymi.

\textbf{Przykład}: sfera $S^n = \{(x_1, \dots, x_{n+1}) \in \mathbb{R}^{n+1} : \sum_{i=1}^{n+1} x_i^2=1\}$ z topologią dziedziczoną z $\mathbb{R}^{n+1}$ jest rozmaitością topologiczną. Pierwsze dwa warunki definiujące rozmaitość topologiczną są dziedziczone z $\mathbb{R}^{n+1}$. Opiszemy rodzinę map takich, że rodzina zbiorów mapowych okryje całą $S^n$.

Dla $i \in \{1, 2, ..., n+1\}$ definiujemy $U_i^+ = \{x \in S^n : x_i > 0\}$, $U_i^- = \{x \in S^n : x_i < 0\}$. Rodzina zbiorów $\{U_i^+, U_i^- : i \in \{1, \dots, n+1\}\}$ stanowi pokrycie $S^n$, ponieważ każdy $x \in S^n$ ma przynajmniej jedną niezerową współrzędną. Odwzorowania $\phi_i^{\pm} \to \mathbb{R}^n$ dane wzorami $\phi_i^{\pm}(x_1, \dots, x_{n+1}) = (x_1, \dots, \hat{x_i}, \dots, x_{n+1})$ (gdzie $\hat{x_i}$ oznacza pominięcie $x_i$) są homeomorfizmami.

Rzeczywiście, obraz tego odwzorowania $\phi_i^{\pm}(U_i^{\pm}) = \{(x_1, \dots, x_n) \in \mathbb{R}^n : \sum_{i=1}^{n} x_i^2 < 1\} = D^n$ stanowi otwarty dysk o promieniu 1 w $\mathbb{R}^n$. Możemy w prosty sposób zdefiniować odwzorowanie odwrotne:
\begin{align*}
	\left(\phi_i^{\pm}\right)^{-1}(x_1, \dots, x_n) = \left(x_1, \dots, x_{i-1}, \pm \sqrt{1-\sum_{i=1}^{n} x_i^2}, x_{i}, \dots, x_n\right).
\end{align*}
Tak określone odwzorowanie jest ciągłe i łatwo można sprawdzić, że jest ono rzeczywiście odwzorowaniem odwrotnym. Zatem funkcje $\phi_{i}^{\pm} : U_{i}^{\pm} \to \bar{U_{i}^{\pm}}$ są homeomorfizmami.

\section{Rozmaitości różniczkowalne (gładkie)}
\subsubsection{Motywacja}
Jeżeli mamy daną funkcję $f : M \to \mathbb{R}$ naturalnym pytaniem jest, czy i w jaki sposób możemy zdefiniować jej różniczkowalność? Pytanie to pojawić się może podczas rozważań natury fizycznej, a także podczas badania różnych obiektów w rozmaitych dziedzinach matematyki, dlatego warto spróbować - do tego właśnie będziemy dążyć. Skoro umiemy dobrze różniczkować funkcje pomiędzy przestrzeniami lokalnie euklidesowymi, pierwszym krokiem jest wyrażenie $f$ w mapie $(U,\phi)$ na $M$, to znaczy badać złożenie $f\circ \phi^{-1} : \mathbb{R}^n \supset \bar{U} \to \mathbb{R}$. Powinniśmy jednak założyć kilka warunków. W przeciwnym wypadku mogłoby dojść na przykład do takiej sytuacji, że w jednej mapie $(U,\phi)$ złożenie $f \circ \phi^{-1}$ było różniczkowane, ale w mapie $(U,\psi)$ mielibyśmy nieróżniczkowalną $f \circ \psi^{-1}$, wówczas mielibyśmy problem. Z tego względu najpierw zadbamy o to, aby wyposażyć rozmaitość w dodatkową strukturę, która pomoże nam wprowadzić różniczkowalność funkcji określonej na rozmaitości.

\subsection{Mapy i atlasy - przygotowanie struktury gładkiej}
\begin{definition}[(Gładka) Zgodność map]
	Mapy $(U,\phi)$ oraz $(U,\psi)$ nazywamy \textbf{zgodnymi} (dokładniej: gładko zgodnymi) jeżeli \textbf{odwzorowania przejścia} $\phi \circ \psi^{-1}$ oraz $\psi \circ \phi^{-1}$ są gładkie, tj. różniczkowalne dowolną ilość razy w każdym punkcie.
\end{definition}

\begin{definition}[(Gładka) Zgodność map]
	Mapy $(U,\phi)$ oraz $(V,\psi)$ nazywamy \textbf{zgodnymi} jeżeli zachodzi jeden z dwóch warunków:
	\begin{itemize}
		\item $U\cap V = \emptyset$,
		\item $U\cap V \neq \emptyset$, wtedy $\phi \big|_{U\cap V}$ oraz  $\psi \big|_{U\cap V}$ są zgodne w sensie poprzedniej definicji.
	\end{itemize}
\end{definition}

Odnotujmy, że zachodzi $f \circ \phi_1^{-1} = \left(f \circ \phi_2^{-1}) \circ \left( \phi_2 \circ \phi_1^{-1} \right)$, tzn. znając opis $f$ w terminach jednej mapy, możemy wyrazić $f$ w terminach drugiej mapy poprzez zastosowanie odpowiedniego odwzorowania przejścia.

\subsubsection{Gładki atlas}
\begin{definition}[Gładki atlas]
	\textbf{Gładkim atlasem} na rozmaitości topologicznej $M$ nazywamy rodzinę map $\left\{(U_{\alpha}, \phi_{\alpha})\right\}_{\alpha}$ takich, że
	\begin{itemize}
		\item $\{U_{\alpha}\}_{\alpha}$ pokrywa $M$ oraz
		\item każde dwie mapy z tego zbioru są gładko zgodne.
	\end{itemize}
\end{definition}

\textbf{Przykład}: rodzina map $\left\{ \left\( U_i^{\pm}, \phi_i^{\pm} \right) : i \in \{1, \dots, n+1 \} \right\}$ na $S^{n+1}$ jest gładko zgodna i zbiory mapowe pokrywają całe $S^{n+1}$, stanowią więc atlas. Sprawdzenie gładkiej zgodności pozostawiamy jako nietrudne ćwiczenie dla czytelnika.

\subsubsection{Uwagi o odwzorowaniu przejścia}
\begin{itemize}
	\item Jeżeli mamy $\phi_1 : U \to \bar{U_1}$, $\phi_2 U \to \bar{U_2}$, to odwzorowanie przejścia $\phi_2 \circ \phi_1^{-1}$ przeprowadza $\bar{U_1}$ w $\bar{U_2}$, natomiast złożenie odwrotne $\phi_1 \circ \phi_2^{-1}$ przeprowadza $\bar{U_2}$ w $\bar{U_1}$.
\end{itemize}
Załóżmy, że mamy dane $(U,\phi)$ oraz $(V,\psi)$. Wówczas:
\begin{itemize}
	\item Odzworowanie przejścia $\phi \circ \psi^{-1}$ będziemy oznaczać $\phi\psi^{-1}$. Mamy zatem $\phi\psi^{-1} : \psi(U\capV) \to \phi(U\capV)$ oraz $\psi\phi^{-1} : \phi(U\cap V) \to \psi(U \cap V)$. Odwzorowania te są homeomorfizmami.
	\item Warto zauważyć, że tak jak byśmy oczekiwali, $\left(\phi\psi^{-1}\right)^{-1} = \psi\phi^{-1}$.
	\item Gdy mapy są gładkie, to odwzorowania przejścia też są gładkie i gładko odwracalne. Tego typu odwzorowania (tj. gładkie i gładko odwracalne) nazywamy \textbf{dyfeomorfizmami}, w tym przypadku mamy dyfeomorfizm między $\psi(U\cap V)$ a $\phi(U \cap V)$.
	\item Dla map z gładkiego atlasu odzworowania przejścia mają w każdym punkcie nieosobliwy Jakobian (macierz pochodnych cząstkowych).
	\item Dla zgodnych map $(U,\phi)$ oraz $(V,\psi)$ i dla $f : M \to \mathbb{R}$ odwzorowanie $f \circ \phi^{-1} \big|_{\phi(U \cap V)}$ jest gładkie wtedy i tylko wtedy gdy gładkie jest odwzorowanie $f \circ \psi^{-1} \big|_{\psi(U \cap V)}$.
\end{itemize}
Jesteśmy już gotowi do wprowadzenia nowej, istotnej struktury dla dalszego badania rozmaitości.

\subsection{Pojęcie rozmaitości gładkiej}
\begin{definition}[Rozmaitość gładka] 
	Parę $(M,A)$ gdzie $M$ jest rozmaitością topologiczną, zaś $A$ jest gładkim atlasem na $M$, nazywamy \textbf{rozmaitością gładką}.
\end{definition}

Pierwszym pytaniem jakie się nasuwa jest to, jak mogą się mieć do siebie dwa atlasy? Jak mogą one względem siebie wyglądać? Czy możemy je jakoś klasyfikować? W odpowiedzi na te pytania musimy wprowadzić kilka dodatkowych pojęć.

\begin{definition}[Zgodność atlasów]
	Mówimy, że \textbf{mapa} $(U, \phi)$ jest \textbf{zgodna z atlasem} $A$, jeżeli jest ona zgodna z każdą mapą $(V,\psi) \in A$. Mówimy, że \textbf{atlasy} $A_1, A_2$ są \textbf{zgodne}, jeżeli każda mapa z $A_1$ jest zgodna z $A_2$.
\end{definition}
Odnotujmy, że zgodne atlasy zadają tę samą strukturę rozmaitości gładkiej na rozmaitości topologicznej $M$.

\begin{definition}[Atlas maksymalny]
	Atlas $A$ nazywamy \textbf{maksymalnym}, jeżeli każda mapa zgodna z $A$ należy do $A$.
\end{definition}

Prawdziwe są następujące dwa fakty, które pomagają nam w pewien sposób klasyfikować atlasy zadające taką samą strukturę gładką na rozmaitości. Ich dowód pozostawiamy jako ćwiczenie.
\begin{itemize}
	\item Każdy atlas $A$ na rozmaitości $M$ zawiera się w dokładnie jednym atlasie maksymalnym na $M$, co więcej jest to atlas złożony z wszystkich map zgodnych z $A$.
	\item Zgodne atlasy zawierają się w tym samym atlasie maksymalnym.
\end{itemize}


