\section{Motywacja}


Motywacją do rozważania równań różniczkowych będzie zagadnienie rozpadu promieniotwórczego.
Eksperymentalnie sprawdzono, że tempo rozpadu jest wprost proporcjonalne do masy pierwiastka promieniotwórczego.
Czyli dla pewnej $k>0$ mamy $x'(t)=-kx(t)$, gdzie $x\left(t \right) $ to masa pierwiastka w chwili $t$.
Szukamy funkcji $m$, pomijać będziemy oznaczenie argumentu, oraz będziemy pisać $\dot{x}:=m'$. 

Zgadnąć możemy, że rozwiązaniem są funkcje postaci $x\left( t \right) = e^{-kt}\cdot x\left( 0 \right) $. Powstaje naturalne pytanie -- czy wszystkie rozwiązania są takiej postaci? W tym przypadku możemy wykonać przejścia równoważne:
\begin{align*}
	\dot{x}+kx&=0\\
	\dot{x}e^{kt}+ke^{kt}x&=0\\
	\left( xe^{kt} \right) ^{\cdot }&=0 
.\end{align*} Zatem $xe^{kt}$ jest stałą, więc $x\left( t \right) =c\cdot e^{-kt}$. 

\section{Definicja}
Układem równań równiczkowych zwyczajnych $m$-tego rzędu nazywamy wyrażenie: \[
	F(\dot{x},\ddot{x},\ldots,x^{(m)})=0 
,\] gdzie 
\begin{align*}
	x: & \quad\mathbb{R}\supseteq I \to \mathbb{R}^{n}\\
	F:&  \quad\mathbb{R}^{1+\left( m+1 \right) n} \to \mathbb{R}^{k}
.\end{align*}

Zazwyczaj udaje się wyrazić najmniejszą pochodną w postaci rozwikłanej: $x^{\left( m \right) }=f\left( t, \dot{x}, \ldots, x^{\left( m-1 \right) } \right) $. Zwykle zakłąda się różniczkowalność lub lipschitzowskość funkcji $f$. 

\section{Redukcja do układu równań pierwszego rzędu}

Jeżeli mamy $x^{\left( m \right) }=f(t,\dot{x},\ldots,x^{\left( m-1 \right) }$, to położywszy $y_{k}=x^{(k)}$, otrzymamy $$\dot{y_{0}} = y_1, \ldots, \quad \dot{y_{m-2}}=y_{m-1}, \quad \dot{y_{m-1}}=y_m=x^{\left( m \right) }=f\left( t,y \right) $$ dla $y=\left( y_0,...,y_{m-1} \right) \in \mathbb{R}^{m}$ . 
A zatem mamy $\dot{y} = g\left( t,y \right) $.
\section{Badanie roztworów nasyconych}
Kolejny przykład zastosowania równań różniczkowych. Stwierdzono, że w cieczy można rozpuścić ilość soli, która zwiększa się proporcjonalnie do zmiany temperatur: $\delta s=k\delta T$, a więc $\frac{\delta s}{\delta T}=kS$, czyli w granicy $\frac{dS}{dt}=kS$. Znamy rozwiązanie tego równania, $S\left( T \right) =S\left( 0 \right) $. Biorąc $s_0=S(T_0), s_1=S(T_1)$, dostaniemy  
